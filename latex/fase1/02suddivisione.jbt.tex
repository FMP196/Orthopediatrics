\textsc{Allegati}: \texttt{analisi\_rischi.ods}. Contiene il computo per la valutazione
dello sforzo.

\section{Capitolato}
 I reparti ortopedia e pediatria di un ente ospedaliero hanno bisogno di una
 applicazione che gestisse le prenotazioni dei pazienti. Il programma sara'
 utilizzato da pazienti e amministratori che avranno maggiori permessi di
 manipolazione sulle prenotazioni.
 
 Le possibili operazioni per gli amministratori dovranno essere:
\begin{itemize}
\item Autenticazione
\item Scelta del reparto
\item Definire prenotazioni ad alta priorità
\item Visualizzare ed aggiornare referti medici
\item Visualizzare lo storico prenotazioni del paziente, e le date disponibili.
\item Stampa
\end{itemize}
 Le possibili operazioni per i pazienti dovranno essere:
\begin{itemize}

\item Registrazione/Autenticazione
\item Scelta del reparto
\item Effettuare una prenotazione
\item Visualizzare lo storico prenotazioni e date disponibili
\item Stampa 
\end{itemize}
\section{Individuazione dei compiti (Risorse umane)}
All'interno di questa fase, si assegnano i ruoli ai seguenti componenti del
team \textit{XYZT}
\begin{description}
\item[Project Manager] Matej Torok
\item[Quality Manager] Giacomo Bergami
\item[Tool Specialist] Paolo de Luca
\item[Developer]       Fabian Priftaj
\end{description}

Il committente di questo progetto è stato il gruppo ``CrySoft''.

\section{Scadenze}
In quanto questa è la nostra prima esperienza di sviluppo di programmazione in ambito
ospedaliero, non possediamo uno storico atto a prefissare una scadenza entro le
quali stabilire delle milestone(s): poiché
nella programmazione extreme programming, si è abituati a seguire passo passo 
l'evoluzione del progetto, punteremo man mano ad aggiornare il cliente ed effettuare
piccoli rilasci successivi di codice funzionante.

Elenchiamo qui sotto le scadenze:
\begin{description}
\item[Piano di Processo] 21-3-2012
\item[Analisi dei Requisiti] 30-03-2012
\item[Revisione al cliente dell'Analisi dei requisiti] 13-04-2012
\item[Rilascio del prodotto al cliente e relazione finale] 25-05-2012
\end{description}
Per il momento è noto che il cliente potrà valutare il prodotto entro il 
\textbf{31 Maggio 2012}.

\section{Valutazione dello sforzo}\label{sec:mfsvds}
Abbiamo la necessità
di effettuare una prima valutazione approssimativa dello sforzo che ci verrà richiesto
ai fini del svolgimento di questo progetto: effettuiamo quindi una stima dei
\textbf{function point}, tramite i \textit{Unadjusted Function Point} (UFP) definiti come
seguono:
\begin{itemize}
\diam Input (EI)
\diam Interrogazioni (EQ)
\diam Output (EO)
\diam File logici interni (ILF)
\diam File logici esterni (EIF)
\end{itemize}

\begin{table}

\centering
\begin{tabular}{cccc}
\toprule
\textbf{Unadjusted Function Point (UFP)} & \textbf{Semplice} & \textbf{Medio} & \textbf {Arduo}\\
\midrule
\textit{Input (EI)} & 3 & 4 & 6\\
\textit{Output (EO)} & 4 & 5 & 7\\
\textit{Interrogazioni (EQ)}& 3 & 4 & 6\\
\textit{File logici interni (ILF)} & 5 & 7 & 10\\
\textit{File logici esterni (EIF)} & 7 & 10 & 15\\
\bottomrule

\end{tabular}
\caption{\textit{Definizione degli} Unadjusted Function Point \textit{ con
definizione dei pesi in base al grado di difficoltà}.}
\label{tab:tdd}

\end{table}

\begin{table}

\centering
\begin{tabular}{cccc}
\toprule
\textbf{Unadjusted Function Point (UFP)} & \textbf{Tipologia} & \textbf{Quantità}\\
\midrule
\textit{Input (EI)} & Semplice & 2\\
\textit{Input (EI)} & Media & 6\\
\textit{Output (EO)} & Semplice & 3\\
\textit{Output (EO)} & Media & 2\\
\textit{Interrogazioni (EQ)} & Semplice & 8\\
\textit{File logici interni (ILF)} & - & 0\\
\textit{File logici esterni (EIF)} & Semplice & 1\\
\bottomrule
Totale & & 75\\
\end{tabular}
\caption{\textit{Definizione degli} Unadjusted Function Point \textit{ con
definizione dei pesi in base al grado di difficoltà}.}
\label{tab:tdd2}

\end{table}

\begin{table}

\centering
\begin{tabular}{cc}
\toprule
\textbf{VAF} & Valutazione scala 1-5\\
\midrule
F1 & 2\\
F2 & 2\\
F3 & 4\\
F4 & 1\\
F5 & 0\\
F6 & 5\\
F7 & 3\\
F8 & 5\\
F9 & 1\\
F10 & 0\\
F11 & 3\\
F12 & 0\\
F13 & 0\\
F14 & 0\\
\bottomrule
Totale & 26\\
\end{tabular}
\caption{\textit{Definizione degli} Unadjusted Function Point \textit{ con
definizione dei pesi in base al grado di difficoltà}.}
\label{tab:tdd3}

\end{table}

In Tabella ~\vref{tab:tdd} riportiamo la valutazione dei pesi assegnati ai precedenti capisaldi
in base alla loro difficoltà\footnote{\texttt{http://www.devshed.com/c/a/Practices/An-Overview-of-Function-Point-Analysis/3/}}. In Tabella  ~\vref{tab:tdd2} riportiamo conseguentemente
i valori da noi utilizzati per effettuare la nostra stima: in quanto non possediamo
ancora informazioni precise rispetto alle esigenze del cliente in merito alla 
 navigabilità e l'usabilità del prodotto, ci basiamo unicamente sulle informazioni
 deducibili dal capitolato pervenutoci. Per ogni \textit{Value
Adjustment Factors} (VAF), la cui lista viene proposta qui sotto, viene attribuito
un \textit{Degree of Influence} (DI), ottenendo la situazione riportata nella Tabella
\vref{tab:tdd3}.

\begin{enumerate}
\item Il sistema ha bisogno di operazioni di backup e di ripristino affidabili?
\item È necessario utilizzare comunicazioni specializzate per trasferire informazioni da o verso l'applicazione?
\item Esistono funzioni di elaborazione distribuite?
\item Le prestazioni rappresentano un fattore critico?
\item Il sistema dovrà funzionare in un ambiente preesistente, pesantemente utilizzato?
\item Il sistema richiede di inserire on-line dei dati?
\item Il sistema on-line richiede una transazione di input costituita da più schermate od operazioni?
\item I files IFL vengono aggiornati on-line?
\item Esistono input/output, file od interrogazioni complesse?
\item L'elaborazione interna è complessa?
\item Il codice è progettato per essere riusabile?
\item Nel progetto sono presenti la conversione e l'installazione?
\item Il progetto è stato concepito per essere installato in più organizzazioni?
\item L'applicazione è progettata per facilitare la modifica e l'usabilità da parte dei clienti?
\end{enumerate}

Possiamo in seguito effettuare il seguente computo:
\[FP = conteggio-UFP \cdot (0.65+0.01\cdot\sum_{i=1}^{14} F_i)\]
Ottenendo quindi
\[FP = 75\cdot (0.65+0.01\cdot 26)=68.25\]
Considerando inoltre che il fattore di conversione FP/LOC in base a Java, che è il
linguaggio di implementazione del prodotto, è $fploc_{Java}=63$ come valore
medio\footnote{R.S. Pressman: ``Principi di Ingegneria del Software''. McGraw-Hill
editore.}, otterremo il seguente valore in LOC
\[LOC = 63\cdot 66=4300\]
Utilizzando inoltre la formula di \textit{Bailey-Basili}\footnote{R.S. Pressman: op. cit.} per il calcolo dello sforzo, che è
\[E_{bb} = 5,5+0,7\cdot {KLOC}^{1,16} mesi/persona\]
Otteniamo
\[E_{bb} = 5.5+0.7\cdot {4,3}^{1.16}\simeq 9.30 mesi/persona\]

Bisogna notare che siamo giunti a questo risultato non conoscendo nulla 
sulle esigenze del cliente, e quindi non conoscendo le user stories, come 
imposto dall'XP. Ci rimandiamo pertanto alla discussione diretta con il cliente
per effettuare un miglioramento della stima iniziale. Confidiamo nel fatto che,
quella fornita, sia una sovrastima allo sforzo complessivo che ci verrà richiesto,
e che riusciremo quindi meglio a valutare solo in seguito all'analisi delle 
\textit{user stories} che ci dovranno essere fornite dai clienti.

\section{Analisi dei rischi}\label{sec:adr}
A questo punto del progetto, non possiamo che fornire un'analisi molto generica
dei rischi dello stesso.

Il team è coeso e motivato a portare a termine il progetto, conseguentemente
non ci dovrebbero essere né problemi di coesione con il team, né di relazioni
personali fra le componenti del gruppo, in quanto tutti i membri del team hanno
affrontato precedentemente altre collaborazioni. 

Inoltre, in quanto XP si basa sulle tecniche di \textbf{planning game}, non ci dovrebbero
essere particolari problemi di scheduling, in quanto giorno per giorno verrà
stabilito il da farsi. 

I rischi che si possono riscontrare ora sono i due seguenti:
\begin{description}
\item[Ambiente di sviluppo] Non tutti i componenti del gruppo conoscono IDE di
	sviluppo quali \textit{Eclipse}: sarà quindi necessario un iniziale costo
	sulla formazione del personale nella conoscenza dei tool da utilizzare.
	Tuttavia, le esperienze precedenti di programmazione e la conoscenza
	di altri linguaggi di programmazione ObjectOriented, dovrebbero semplificare
	la fase di apprendimento.
\item[Cliente] La maggiore incognita a questo punto del progetto consta proprio
	nella figura del Cliente, che non abbiamo ancora avuto l'occasione di
	incontrare. In questo caso il rischio è dovuto all'impossibilità di
	valutare i seguenti capisaldi:
\begin{itemize}
\diam Qual è il grado di sofisticazione del cliente, in quale modo si è in
	grado di comunicare con lo stesso, e con quale tempestività?
\diam I requisiti sono stati ben compresi dal team di sviluppo e dai clienti?
\diam Gli utenti finali hanno attese realistiche?
\end{itemize}
\end{description}

\section{Diario}
\begin{description}
\item[Giovedì 8 Marzo] L'intero gruppo si è riunito nel pomeriggio per discutere 
	la pianificazione dei giorni successivi. Durante l'incontro si è discusso
	per:
\begin{enumerate}
\item l'assegnamento dei ruoli come precedentemente definito;
\item valutazione delle componenti in gioco nell'analisi dei rischi e sul calcolo
	dello sforzo;
\item identificazione dei primi dubbi sulle specifiche da chiarire con il cliente.
	(verranno dettagliati con dovizia di particolari nella fase successiva 
	del progetto);
\item identificazione cenni all'utilizzo dei tool;
\item organizzazione sulle modalità di contatto per reperire i clienti (a noi 
	non familiari)
\item pianificazione del successivo meeting;
\item prima stesura del documento.
\end{enumerate}
Totale ore di lavoro: 6
\item[Venerdì 9 Marzo] Durante la sera il gruppo si è riunito ``virtualmente'' 
	su Skype. Durante l'incontro si è discusso per:
\begin{enumerate}
\item la presentazione della prima stesura del \textit{Piano di processo} da parte del Quality Manager:
	da questa è nata un'accesa discussione riguardo i Value Adjustment Factors
	ed in seguito si è valutata la miglior formula per il calcolo dei FP;
\item le ipotetiche richieste del cliente 
	come qualità, diversificazione e semplicità di input; si prevede un 
	primo incontro con il cliente per il dì \textbf{23 Marzo 2012}
\item la definizione dei tool, e raffinamento delle scadenze, già precedentemente discusse.
\item seconda ed ultima stesura del documento;
\end{enumerate}
Totale ore di lavoro: 6
\end{description}
