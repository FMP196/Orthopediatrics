\part{Documento Terzo}
\chapter{Analisi I}\label{cha:pds}
\minitoc\mtcskip
\section{Premessa}
La fase di sviluppo si strutturerà come segue:
\begin{itemize}
\diam \textit{Definizione del Modello di Progetto}

Partendo dal \textit{Glossario} (Sezione \vref{sec:glossario}) definito nella fase 
precedente, si %% TODO: link
definiscono le prime schede CRC (v. Sezione\vref{sec:crccards}), dalle quali si 
arriva  alla definizione del \textit{Modello di Dominio} del nostro applicativo 
(v. Sezione \vref{sec:domainmodel}); da quest'ultimo discende pure
l'\textit{Analisi del Database}.
\medskip

\diam \textit{Documento dell'architettura software}

Si effettua una prima revisione del modello di dominio, iniziando con l'analisi
delle operazioni fondamentali che saranno richieste dall'\textsc{Utente} del sistema
per conto del \textsc{Paziente} (sè stesso o il suo tutelato).
\medskip

Partendo dalla trattazione dei Casi d'Uso, si effettua la descrizione 
dell'architettura software mediante la \textbf{Vista dei Dati} mediante \textsc{Diagrammi
di Attività}, e la \textbf{Vista dei Casi d'Uso} tramite \textsc{Diagrammi di Sequenza}.
Questi verranno successivamente raffinati, una volta completato il \textit{Modello
di Business} dell'applicazione.

\end{itemize}
Associamo inoltre i seguenti documenti:
\begin{itemize}
	\item \textit{Analisi del Database}:
	(v. Capitolo \vref{cha:progdbasdata}) in seguito alla definizione del
	modello di progetto, viene effettuata la struttureazione del database,
	in modo da ottenere già le componenti base sulle quali effettuare.
	\item \textit{Analisi del Client}:
	(v. Capitolo \vref{sec:clientanalisys}) ci limitiamo a descrivere le
	operazioni che possono essere effettuate tramite l'interfaccia grafica.
	\item \textit{Analisi del Server}:
	(v. Capitolo \vref{sec:serveranalisys}) anche in questo caso, ci limitiamo 
	a descrivere le operazioni che possono essere effettuate tramite 
	l'interfaccia grafica.
\end{itemize}
\bigskip

