\chapter{Revisioni delle stime precedenti}
\minitoc\mtcskip

\section{Ulteriore estensione dei Casi d'Uso}
Facendo riferimento al Testo dei Casi d'Uso della Sottosezione \vref{subsec:usecasetext},
dobbiamo sottolineare la necessità di effettuare delle estensioni al caso d'uso
\textit{Registrare il paziente}. Per completezza, riportiamo qui l'intero testo
del caso in questione con le modifiche apportate, utilizzando sempre lo stile espositivo
proposto dal Larman.


\begin{description}
\item[Nome del caso d'uso] Registrare il paziente
\item[Attore primario]
	L'utente (del sistema): il paziente maggiorenne od il tutore del paziente minorenne

\item[Parti interessate ed interessi]
	L'utente: vuole registrare o sé stesso, o il paziente minorenne di cui ha tutela

\item[Pre-condizioni]
	Nessuna

\item[Garanzia di successo]
	Il sistema visualizzerà all'utente un messaggio di avvenuta registrazione

\item[Scenario principale di successo]

\begin{enumerate}
	\item L'utente sceglie l'opzione di registrazione
	\item L'utente inserisce i dati del paziente che dovrà ricevere le cure
	\item L'utente richiede la registrazione
	\item L'utente riceve la notifica di avvenuta registrazione con una notifica tramite il sistema
\end{enumerate}

\item[Estensioni]
\begin{description}
	\item[Estensione A]
	\medskip
	
	\begin{description}
	\item[4] L'utente richiede la registrazione con successo in quanto è un paziente maggiorenne
	\end{description}
\end{description}


\begin{description}
	\item[Estensione B]
	\medskip
	
	\begin{description}
	\item[4] 
	\begin{enumerate}
		\item Il sistema richiede all'utente l'associazione delle credenziali del tutore, in quanto il paziente in fase di registrazione è minorenne
		\item Il tutore inserisce le proprie credenziali
		\item Il sistema conferma la presenza del tutore all'interno del sistema 
		\item Il sistema effettua la registrazione del paziente minorenne e dell'associazione con il tutore utente del sistema
	\end{enumerate}
	\end{description}
\end{description}

\begin{description}
	\item[Estensione C]
	\medskip
	
	\begin{description}
	\item[4] 
	\medskip
	
	Il sistema informa l'utente della mancata registrazione del paziente, specificando quali siano i campi che sono stati ignorati in fase di registrazione
	\end{description}
\end{description}

\begin{description}
	\item[Estensione D]
	\medskip
	
	\begin{description}
	\item[4B (3)] 
	\medskip
	
	\begin{enumerate}
	\item Il sistema informa il tutore che lui stesso non esiste all'interno del sistema
	\item Il sistema annulla la registrazione del minore 
	\end{enumerate}
	\end{description}

\end{description}


\begin{description}
	\item[Estensione E]
	\medskip
	
	\begin{description}
	\item[2] 
	\medskip
	
	\begin{enumerate}
	\item In quanto il paziente in questione che sta effettuando la registrazione
		è già presente all'interno del sistema come tutore (\textit{precondizione}),
		ed è quindi maggiorenne, richiede di registrarsi come tutore
		importando quei dati.
	\item Si autentica come Tutore nel sistema
	\item L'utente riceve una notifica di conferma dal sistema
	\end{enumerate}
	\end{description}

\end{description}

\begin{description}
	\item[Estensione F]
	\medskip
	
	\begin{description}
	\item[B2] 
	\medskip
	
	\begin{enumerate}
	\item Il nuovo tutore inserisce le proprie credenziali, specificando che esiste
		come Paziente Maggiorenne, autenticandosi come tale.
	\item Il sistema conferma l'esistenza del Paziente Maggiorenne dandone
		notifica
	\item Il sistema crea un nuovo Tutore dalle credenziali del Paziente
		Maggiorenne del quale si è comprovata l'esistenza
	\item Il sistema conferma la registrazione del paziente minorenne
		e vi associa il nuovo tutore
	\end{enumerate}
	\end{description}

\end{description}

\begin{description}
	\item[Estensione G]
	\medskip
	
	\begin{description}
	\item[B2] 
	\medskip
	
	\begin{enumerate}
	\item Il nuovo tutore specifica che vuole essere registrato nel sistema
	\item Il Tutore immette le sue informazioni all'interno del sistema
	\item Il sistema conferma la presenza del nuovo tutore all'interno del sistema 
	\item Il sistema effettua la registrazione del paziente minorenne e 
		dell'associazione con il tutore utente del sistema
	\end{enumerate}
	\end{description}

\end{description}
\item[Requisiti non funzionali]
\begin{itemize}
\diam Il sistema informativo deve poter prevedere autentificazione per i vari
	utenti e per l'amministratore, ed in più il paziente minorenne può accedere solamente mediante
	la supervisione del tutore; è richiesta autenticazione  sia per il
	paziente minorenne, sia per il tutore.
\end{itemize}
\end{description}
\begin{center}
\line(1,0){250}
\end{center}

