\part{Documento Quarto}\label{part:docQuar}
\chapter{Sviluppo I}\label{cha:pds}
\minitoc\mtcskip
\section{Premessa}
In questo documento, cerchiamo di apportare un'ultima e finale modifica al 
modello di dominio, allo scopo di rendere possibile la creazione dei seguenti
documenti, che portano globalmente alla prima stesura di codice

\begin{itemize}
\diam \textit{Raffinamento del modello di dominio}

Questo documento è alla base delle documentazioni che verranno fornite: si vogliono
applicare a quest'ultimo alcuni \textit{Design Pattern}(s), allo scopo di prevenire
successive problematiche di implementazione (v. SottoSottosezione 
\vref{subsubsec:applydesignpatt}).
In seguito, si procederà con l'assegnamento dei metodi alle rispettive classi
tramite l'applicazione dei pattern GRASP (v. Sezione \vref{sec:raffmoddom}).
In questa fase sono già contemplate le considerazioni che emergono dai due punti
successivi.
\medskip

\diam \textit{Analisi del database}

In base al raffinamento del modello di dominio sopra proposto, si effettua
una definizione di un modello ER per la nostra base di dati, e conseguentemente
si produce il codice SQL relativo (v. Capitolo \vref{cha:progdbasdata}).
Tramite l'analisi del database introduciamo per la
prima volta il \textit{framework} EJP all'interno dei nostri elementi di sviluppo,
in base alla quale effettuare il mapping tra modello ER (\textit{Entity-Relationship}) 
ed OO (\textit{Object Oriented}). In base a queste considerazioni, riteniamo
conseguentemente inutile lo sviluppo di un server come proposto nella fase
precedente (v. Sezione \vref{sec:serveranalisys}), in quanto già questo \textit{framework }
risolve a nostro parere, moltissimi problemi di implementazione.
(v. Sezione \vref{sec:dbutilizzejp}).
\medskip



\medskip
\end{itemize}
