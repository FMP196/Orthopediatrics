%% LyX 1.6.5 created this file.  For more info, see http://www.lyx.org/.
%% Do not edit unless you really know what you are doing.
\documentclass[12pt,english,italian]{amsbook}

%\usepackage{savesym}
%\usepackage{makeidx}
%\savesymbol{see}\savesymbol{end}
%\restoresymbols{mammagamma}{see}
%\restoresymbols{mammagamma}{end}

\usepackage[all]{xy}
\usepackage{framed}

\usepackage{lmodern}
\usepackage{tikz-er2}
\usepackage{tikz,xifthen}
\usepackage{pgf}
\usepackage{listings}
\usepackage{courier}
\usepackage{paralist}
\usepackage[T1]{fontenc}
\usepackage[utf8]{inputenc}
\usepackage{braket}
\usepackage[a4paper]{geometry}
% \geometry{verbose,tmargin=4cm,bmargin=4cm,lmargin=1.25cm,rmargin=1.25cm,headheight=2cm,headsep=2cm,footskip=2cm}
\setcounter{secnumdepth}{9}
\setcounter{tocdepth}{9}
\usepackage{color}

\usepackage{mathpazo}
\usepackage[tight,italian]{minitoc}
\usepackage{array}
\usepackage{float}
\usepackage{units}
\usepackage{textcomp}
\usepackage{graphicx}
\usepackage{booktabs}
\usepackage{tabularx}
\usepackage{rotating}
\usepackage[italian]{varioref}
\usepackage{subfig}
\usepackage{ulem}

%% NOTA: makeidx già supportato da ams

%% MATH
\usepackage{amsfonts}
\usepackage{amsmath,amsthm}
\usepackage{amsthm}
\usepackage{mathpazo}
\usepackage{amssymb}
\usepackage{mathdots}
\usepackage{synttree}
\PassOptionsToPackage{normalem}{ulem}
\usepackage{ulem}

%% BIBLIOGRAPHY
\usepackage[italian]{babel}
\usepackage[babel]{csquotes}
\usepackage{biblatex}
\bibliography{bibliografia}

\usepackage[unicode=true, pdfusetitle,
 bookmarks=true,bookmarksnumbered=false,bookmarksopen=false,
 breaklinks=true,pdfborder={0 0 1},backref=false,colorlinks=false]
 {hyperref}

\makeatletter

\newcommand{\noun}[1]{\textsc{#1}}
%% Special footnote code from the package 'stblftnt.sty'
%% Author: Robin Fairbairns -- Last revised Dec 13 1996
\let\SF@@footnote\footnote
\def\footnote{\ifx\protect\@typeset@protect
    \expandafter\SF@@footnote
  \else
    \expandafter\SF@gobble@opt
  \fi
}
\expandafter\def\csname SF@gobble@opt \endcsname{\@ifnextchar[%]
  \SF@gobble@twobracket
  \@gobble
}
\edef\SF@gobble@opt{\noexpand\protect
  \expandafter\noexpand\csname SF@gobble@opt \endcsname}
\def\SF@gobble@twobracket[#1]#2{}
%% Because html converters don't know tabularnewline
\providecommand{\tabularnewline}{\\}
\floatstyle{ruled}
\newfloat{algorithm}{tbp}{loa}
\floatname{algorithm}{Algorithm}

\numberwithin{section}{chapter}
\numberwithin{equation}{section}
\numberwithin{figure}{section}
 %\theoremstyle{definition}
 %\newtheorem{defin}{Definizione}
 \newtheorem*{defn*}{Definizione}
 \newtheorem{defin}{Definizione}
  \newtheorem*{prop*}{Proposizione}
  %\theoremstyle{definition}
  \newtheorem*{problem*}{Problema}
  %\theoremstyle{definition}
  \newtheorem*{example*}{Esempio}
  \newtheorem{example}{Esempio}[chapter]
  \newtheorem{ex}{Esercizio}[chapter]
  \newtheorem*{lem*}{Lemma}
  \newtheorem*{thm*}{Teorema}
\newtheorem{thm}{Teorema}[section]
\newtheorem{coroll}{Corollario}[section]
\newtheorem{theorem}{Teorema}[section]
  \newtheorem{prop}{Proposizione}[section]
  \newtheorem{lem}{Lemma}[section]
  %\theoremstyle{definition}
  \newtheorem{defn}{Definizione}[section]
  

%%%%%%%%%%%%%%%%%%%%%%%%%%%%%% User specified LaTeX commands.
\usepackage{listings}
%\usepackage[usenames]{color}
\usepackage{courier}
 \lstset{
         basicstyle=\footnotesize\ttfamily, % Standardschrift
         %numbers=left,               % Ort der Zeilennummern
         numberstyle=\tiny,          % Stil der Zeilennummern
         %stepnumber=2,               % Abstand zwischen den Zeilennummern
         numbersep=5pt,              % Abstand der Nummern zum Text
         tabsize=2,                  % Groesse von Tabs
         extendedchars=true,         %
         breaklines=true,            % Zeilen werden Umgebrochen
         keywordstyle=\bfseries,
    		frame=b,         
 %        keywordstyle=[1]\textbf,    % Stil der Keywords
 %        keywordstyle=[2]\textbf,    %
 %        keywordstyle=[3]\textbf,    %
 %        keywordstyle=[4]\textbf,   \sqrt{\sqrt{}} %
         stringstyle=\color{white}\ttfamily, % Farbe der String
         showspaces=false,           % Leerzeichen anzeigen ?
         showtabs=false,             % Tabs anzeigen ?
         xleftmargin=17pt,
         framexleftmargin=17pt,
         framexrightmargin=5pt,
         framexbottommargin=4pt,
         %backgroundcolor=\color{lightgray},
         showstringspaces=false      % Leerzeichen in Strings anzeigen ?  
 }
 \lstloadlanguages{% Check Dokumentation for further languages ...
         %[Visual]Basic
         %Pascal
         %C
         %C++
         %XML
         %HTML
         Java
 }
    %\DeclareCaptionFont{blue}{\color{blue}} 

  %\captionsetup[lstlisting]{singlelinecheck=false, labelfont={blue}, textfont={blue}}
  \usepackage{caption}
\DeclareCaptionFont{white}{\color{white}}
\DeclareCaptionFormat{listing}{\colorbox[cmyk]{0.43, 0.35, 0.35,0.01}{\parbox{\textwidth}{\hspace{15pt}#1#2#3}}}
\captionsetup[lstlisting]{format=listing,labelfont=white,textfont=white, singlelinecheck=false, margin=0pt, font={bf,footnotesize}}
\lstdefinelanguage{concurrentg}  {
morekeywords={while, cobegin, coend, do, if, true, false, else, void, int, new, for, shared, let, rec, then, to, match, with, raise, done, begin, end, not, each, in, return},
morecomment=[s][]{/*}{*/},
morestring=[b][\color{red}]"
}
\lstdefinelanguage{Scheme}  {
morekeywords={set,first,cons,make,struct,cond,true,false},
morecomment=[l]{;},
}
\lstset{defaultdialect=[Visual]Basic}
\lstnewenvironment{concurrentg}{\lstset{language=concurrentg}}{}





% mie definizoni 
\newcommand{\diam}{\item[$\diamond$]} % punto elenco diamante
\newcommand{\cerc}{\item[$\circ$]} % punto elenco cerchio
\newcommand{\suspence}{[\dots]} % ellissi
\newcommand{\opquote}{«}
\newcommand{\clquote}{»}
\newcommand{\reffig}{Figura}
\newcommand{\singlediam}[1]{\begin{itemize}\diam #1\end{itemize}} % un solo punto elenco con diamante
\newcommand{\ttilde}{\textasciitilde{}  } % scrittura semplificata della tilde
\renewcommand{\*}{\cdot}
\renewcommand{\d}{\$}
\newcommand{\myquote}[2]{«\emph{#1}»(#2)}
\newcommand{\myquotu}[1]{«\emph{#1}»}
\newcommand{\textbb}[1]{$\mathbb{#1}$}

\newcommand{\nat}{\mathbb N}
\newcommand{\real}{\mathbb R}
\newcommand{\bool}{\mathbb B}
\newcommand{\lang}{\mathcal L}
\newcommand{\down}{\downarrow}
\newcommand{\up}{\uparrow}


%% Form Net
\newsavebox{\sembox}
\newlength{\semwidth}
\newlength{\boxwidth}

\newcommand{\Sem}[1]{%
\sbox{\sembox}{\ensuremath{#1}}%
\settowidth{\semwidth}{\usebox{\sembox}}%
\sbox{\sembox}{\ensuremath{\left[\usebox{\sembox}\right]}}%
\settowidth{\boxwidth}{\usebox{\sembox}}%
\addtolength{\boxwidth}{-\semwidth}%
\left[\hspace{-0.3\boxwidth}%
\usebox{\sembox}%
\hspace{-0.3\boxwidth}\right]%
}

 % definizione dell'elenco con le lettere (wikipedia.en)
%\arabic 	1, 2, 3 ...
%\alph 	a, b, c ...
%\Alph 	A, B, C ...
%\roman 	i, ii, iii ...
%\Roman 	I, II, III ...
%\fnsymbol 	Aimed at footnotes (see below), but prints a sequence of symbols.
 % come consigliato dalla Nasa (http://www.giss.nasa.gov/tools/latex/ltx-222.html):
 % \renewcommand{\labelenumi}{\emph{(\alph{enumi})}}
\newenvironment{itmletters}[1]{\begin{enumerate}\renewcommand{\labelenumi}{\emph{#1}}}
                              {\end{enumerate}}
\newenvironment{stab}{\begin{table}\begin{shaded}}
                              {\end{shaded}\end{table}}
 % questo comando crea l'elenco con i letterali definiti sopra
\newcommand{\atm}{\alph{enumi}}
 % questo comando serve per indicare in quello sopra l'enumerazione della prima serie di numeri
 % come di 
\newcommand{\itm}{\roman{enumi}}
\newcommand{\cent}[1]{\begin{center}#1\end{center}}
\newcommand{\textcal}[1]{$\mathcal{#1}$}
                        
\newcommand{\startfrom}[1]{\setcounter{enumi}{#1}}
\newcommand{\ttitem}[1]{\item \texttt{#1}}
\newcommand{\tleq}{$\leq$} 
\newcommand{\tgeq}{$\geq$}


\newtheorem{prob}{Problema}[section]
\newtheorem{xmpl}{Esempio}[section]
\newtheorem{propr}{Proprietà}[section]

\hypersetup{
 pdfsubject={Relazione finale},
 pdfkeywords={progetto lis relazione xyzt}}

%% ONLY FOR L.I.S PROJECT :)
%% http://student.cosy.sbg.ac.at/~hhofbaue/se1/index.shtml
\newenvironment{CRCcard}[3]{
\begin{tabular}{ll}
\toprule
\multicolumn{2}{l}{\bf{#1}}\\
\textit{Superclasses}: 	& #2\\
\textit{Subclasses}:	& #3\\
\hline
\textit{Responsibilities} &\textit{Collaborators}\\}
{\bottomrule
\end{tabular}}

\newcommand{\rcline}[2]{ \textit{#1}: & \textsc{#2} \\ }	


\makeindex
\makeatother
\dominitoc
